\documentclass{article}
\usepackage{import}
\import{./}{header}

\usepackage{fancyhdr}
\pagestyle{fancy}
\lhead{Web технологии}
\rhead{Введение}
\lfoot{Комплексные числа}
\rfoot{Введение}
%\def\changemargin#1#2{\list{}{\rightmargin#2\leftmargin#1}\item[]}
%\let\endchangemargin=\endlist 
\title{Конспект по комплексным числам}
\usepackage{lipsum}

\begin{document}
\maketitle
\thispagestyle{fancy}
\section{HTTP.}
\df HTTP --- Hypertext Transfer Protocol, гипертекст --- это текст с отсылками на другие тексты, изначально использовался в научных публикациях.

Изначально HTTP --- клиент-сервер.

Сейчас используется версия HTTP 1.1.

Можно заметить, что путь и домен передаются раздельно. Это сделано для того, чтобы быстрее получить домен из запроса и проверить, обсуживается данным сервером этот домен. Дело в том, что сервер может хостить сразу несколько доменов. И таким образом, это делается для того, чтобы уменьшить нагрузку на сервер.

В HTTP-ответе указывается длина ответа в байтах. Для чего это нужно? Дело в том, что лучше знать какое количество данных нам хотят отправить. Иногда клиент не может принять слишком много данных. А во-вторых, можно проверить, пришли ли все данные или нет.

В Content-Type указывается MIMI-type.

HTTP-методы:
\begin{itemize}
	\item GET --- получить что-то с сервера
	\item HEAD --- получить только заголовки для этого документа, например, получить размер видео (которое хотим загрузить с сервера)
	\item POST --- изменение содержимого на сервере, а именно обновление информации
	\item PUT --- изменение содержимого на сервере, а именно добавление цельного куска информации
\end{itemize}
По факту в разработке используется только GET и POST запросы.

HTTP-заголовки:
\begin{itemize}
	\item Host --- домен, к которому обращаемся
	\item Date --- дата
	\item Referer --- указывает, с какой страницы мы кликнули. Используется для сбора статистики.
	\item Content-Length --- длина запроса
	\item Content-Encoding --- кодировка, в которой мы обмениваемся данными
	\item User-Agent --- строка, которая идентифицирует клиент (браузер, смартфон)
\end{itemize}

HTTP-коды ответа:
\begin{itemize}
	\item 1xx --- информационные
	\item 2xx --- успешное выполнение (самый частый --- 200)
	\item 3xx --- перенаправление
	\item 4xx --- ошибка на стороне клиента
	\item 5xx --- ошибка на стороне сервера
\end{itemize}

Accept-заголовки --- заголовки согласования содержимого: языков, кодировки и т.д.

HTTP управление соединением
\begin{itemize}
	\item Connection: close --- получил ответ - рвешь соединение
	\item Connection: keep-alive --- продолжаем общение
\end{itemize}

HTTP-cookie --- дают возможность персонализировать информацию. Используется для авторизации и для отслеживания пользователя.

\end{document}


