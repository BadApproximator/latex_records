\usepackage[utf8]{inputenc}
\usepackage[T2A]{fontenc}
\usepackage[russian]{babel} 

\usepackage{graphicx,amssymb,amsmath,amsfonts,latexsym,mathtext, mathrsfs, amsthm, thmtools}
\usepackage[a4paper,left=2cm, right=2cm, top=2cm, bottom=2cm, bindingoffset=0cm]{geometry}

\usepackage{indentfirst}

\usepackage{setspace}
% полуторный интервал
\onehalfspacing

% \usepackage{pscyr} % Нормальные шрифты
\usepackage{enumerate}

\usepackage[usenames]{color}
\usepackage{colortbl}

\usepackage{floatflt}
\renewcommand{\vec}[1]{\boldsymbol{#1}}
%\usepackage{indentfirst}   

\usepackage{cases}

\usepackage{import}
%-------------------------------------------------------
\usepackage{bbold} % Жирная единица
\usepackage{amscd} % Диаграммы

% для квадратных скобок под формулой--------------------
\usepackage{mathtools}% http://ctan.org/pkg/mathtools

% Для текста в рамочке в качестве отступления/напоминания
\usepackage{xcolor}
\usepackage{mdframed}

% для зачеркивания в формулах
\usepackage{cancel}

\newcommand{\ph}{\varphi}
\newcommand{\ep}{\varepsilon}
\newcommand{\s}{\sigma}
\newcommand{\ws}{\widetilde{\sigma}}
\newcommand{\wmu}{\widetilde{\mu}}
\newcommand{\w}{\widetilde}
\newcommand{\vkappa}{\varkappa}

\renewcommand{\ge}{\geqslant}
\renewcommand{\le}{\leqslant}

\newcommand{\R}{\mathbb{R}}
\newcommand{\Co}{\mathbb{C}}
\newcommand{\Na}{\mathbb{N}}
\newcommand{\la}{\lambda}

\newcommand{\llangle}{\left\langle}
\newcommand{\rrangle}{\right\rangle}
\newcommand{\braces}[1]{\left(#1\right)}
\newcommand{\lrangle}[1]{\left\langle #1 \right\rangle}

\newcommand{\threestars}{\begin{center}$ {\ast}\,{\ast}\,{\ast} $\end{center}}

\DeclareMathOperator{\AC}{AC}
\DeclareMathOperator{\SL}{SL}
\DeclareMathOperator{\Ker}{Ker}
\DeclareMathOperator{\Ima}{Im}
\DeclareMathOperator{\Rea}{Re}

\theoremstyle{definition}
\newtheorem{innercustomthm}{Теорема}
\newenvironment{customthm}[1]
{\renewcommand\theinnercustomthm{#1}\innercustomthm}
{\endinnercustomthm}

\newtheorem{innercustomlm}{Лемма}
\newenvironment{customlm}[1]
{\renewcommand\theinnercustomlm{#1}\innercustomlm}
{\endinnercustomlm}

\newtheorem{innercustomcor}{Следствие}
\newenvironment{customcor}[1]
{\renewcommand\theinnercustomcor{#1}\innercustomcor}
{\endinnercustomcor}

\newcommand{\df}{\par\noindent%
    \textbf{Опр. } }
\newcommand{\notion}{\par\noindent%
    \textbf{Обозначение.} }
\newcommand{\note}{\par\noindent%
    \textbf{Замечание.} }
\newcommand{\state}{\par\noindent%
    \textbf{Утвержение.} }
\newcommand{\cor}{\par\noindent%
    \textbf{Следствие.} }

\newcommand{\ex}{\par\noindent
    \textbf{Пример.} }
\newcommand{\z}{\par\noindent%
    \textbf{Задача.} }

\newdimen\theoremskip
\theoremskip=2pt
\renewenvironment{proof}{\par\noindent$\square\quad$}{$\hfill\blacksquare$ \par\vskip\theoremskip} %hfill for align at the end of line

\newenvironment{smallproof}{\color{blue!50!black}\par\noindent$\triangleright\quad$}{$\hfill\triangleleft$ \par\vskip\theoremskip}

\usepackage{stackengine}
\stackMath
\newcommand\frightarrow{\scalebox{1}[.4]{$\rightarrow$}}
\newcommand\darrow[1][]{\mathrel{\stackon[1pt]{\stackanchor[1pt]{\frightarrow}{\frightarrow}}{\scriptstyle#1}}}